\section{OpenCL - A Multipurpose Programming Model}
\label{sec:OpenCL}
  OpenCL is a standardized programming model similar to CUDA.
  However, OpenCL is developed by multiple hardware vendors and is not limited to GPUs.
  It can be used for several heterogeneous platforms, e.g. FPGAs.
  Therefore, the context has to be explicitly specified.
  This makes OpenCL more complex than CUDA, but also enables more diversity in hardware.~\cite{Rauber.2012}

  OpenCL is very similar to CUDA.
  It also separates a program in host and device program.
  Furthermore, it uses the same kind of memory hierarchy.
  However, when calling a kernel function, the environment of the kernel to execute in has to be defined.
  This includes the devices to run on and kernel objects which are the kernel functions and arguments to be used.
  Additionally, the memory objects have to be defined.
  These are the variables used during the kernel execution.
  The executable of the kernel function has to be created and stored in a program object.~\cite{Khronos.2019}
  
  
  The OpenCL API enables the host to create these contexts and to interact with a device via the \textit{command-queue}.
  Different commands can be put into the queue to execute a kernel, to transfer data or to synchronize execution.
  Each command being put in the queue starts at the \textit{queued} stage.
  In this state it can be flushed explicitly before reaching the other five stages.
  The next states the command passes are \textit{submitted} and \textit{ready}.
  At this point all requirements for the command to be executed are completed.
  The execution finally starts at the \textit{running} stage and the command is assigned to the \textit{ended} state when the execution has finished.
  After all value updates are written to memory the command is of state \textit{complete}.
    
  There are some differences in terminology, too.
  CUDA's threads are called \textit{work-items}, blocks are \textit{work-groups} and grids are named \textit{global NDRanges}.
  Additionally it introduces new objects, e.g. \texttt{cl\_mem} which is used for device memory.
    
  Furthermore, in OpenCL kernel functions are not defined in the main file.
  They are defined in additional files which are read line by line and created by the function \texttt{clCreateProgramWithSource()}.
  The output is then built by \texttt{clBuildProgram()}.
  Another difference which makes a OpenCL program more complex than it's CUDA counterpart, is that OpenCL has to create the context defining the device information.
  Additionally, each parameter of a kernel function has to be passed one by one via \texttt{clSetKernelArg()} and the kernel call has to be queued in the command queue of the device.
