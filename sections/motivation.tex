\section{Motivation}
  Graphics processing units (GPUs) are increasingly used for different fields of computing they were created originally.
  Instead of accelerating graphical computations, they are more and more used for e.g. machine learning and mining cryptographic currencies.
  This phenomenon is called \textit{General Purpose} GPU programming.~\cite{8363085}~\cite{Owens.2008}
  
    According to the authors of \cite{NVIDIA.2019} and \cite{Rauber.2012} GPUs are specialized for highly parallel computation.
    They developed rather independent from the central processing unit (CPU) and focused on graphical processing.
    This field provides huge amounts of data to process without internal dependencies.
    Which means, the data can be processed simultaneously.
    Thereby GPUs nowadays have significantly higher compute power than CPUs do, in case the data allows the GPU to benefit from the parallel execution.
    
    The aim of this paper is to provide an introduction to GPU programming.
    The next chapter explains the architecture and design principles of GPUs and compare them to CPU architecture.
    Afterwards CUDA is introduced and the mapping of a program to the hardware is explained.
    Section \ref{sec:OpenCL} then concentrates on OpenCL and states its differences to CUDA.
    An example matrix multiplication is implemented in the last chapter using the findings and learnings of this paper.