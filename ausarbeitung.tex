% This is LLNCS.DEM the demonstration file of
% the LaTeX macro package from Springer-Verlag
\documentclass[a4paper,12pt]{llncs}
%
\usepackage{makeidx}  % allows for indexgeneration
\makeindex

\usepackage[ngerman]{babel}
\usepackage[utf8]{inputenc}      % Code-Page latin 1
\usepackage[T1]{fontenc}
\usepackage{listings}
% Nur eine der beiden folgenden Zeilen einbinden!
% siehe Abschnitt Bilder
%\usepackage{graphicx}       % Bilder einbinden, Version fuer normales latex
\usepackage[pdftex]{graphicx}       % Bilder einbinden, Version fuer pdflatex

% mit Hyperrefs
\usepackage[pdftex, plainpages=false,hypertexnames=true,pdfnewwindow=true,backref=true,colorlinks=true,citecolor=blue,linkcolor=black,urlcolor=blue,filecolor=blue]{hyperref}%
% weitere Packages
\usepackage{ifthen}                 % Zum Auskommentieren von Textteilen
\usepackage{amssymb}                % Mathematische Buchstaben
\usepackage{amsmath}                % Verbesserter Formelsatz
\usepackage{booktabs}               % schönere Tabellen
\usepackage{color}
\usepackage{hyperref}
 \hypersetup{urlcolor=black,citecolor=black}
\usepackage{dsfont}
%\newtheorem{definition}{Definition}
\usepackage{doc}

% Seitenformat ===============================================================
\hoffset=-1.25truecm
\setlength{\topmargin}{0.0cm}
\setlength{\textheight}{23.0cm}
\setlength{\footskip}{1.5cm}
\setlength{\textwidth}{15.4cm}
\setlength{\evensidemargin}{1.5cm}
\setlength{\oddsidemargin}{1.5cm}
\setlength{\parskip}{1ex}
\setlength{\parindent}{0pt}
\setlength{\marginparwidth}{1.4cm}
\setlength{\marginparsep}{1mm}

\pagestyle{plain}

% LstListing-Format ==========================================================
\lstdefinestyle{cpp}{
  language=C++,
  basicstyle=\small\ttfamily,
  frame=tb,
  xleftmargin=\parindent,
  keywordstyle=\color{blue},
  stringstyle=\color{red},
  commentstyle=\color{green},
  morecomment=[l][\color{magenta}]{\#},
  framexleftmargin=5pt,
  framexrightmargin=5pt,
  framextopmargin=5pt,
  framexbottommargin=5pt,
  literate={~}{$\sim$}1
}

% Makro-Definitionen ==========================================================
% Zahlenbereiche -------------------------------------------------------------
\newcommand{\N}{{\mathbb{N}}}
\newcommand{\R}{{\mathbb{R}}}
\newcommand{\C}{{\mathbb{C}}}
\newcommand{\Z}{{\mathbb{Z}}}
\newcommand{\Q}{{\mathbb{Q}}}

%
\def\myverzeichnis{.}

\numberwithin{equation}{section}
% Bild -----------------------------------------------------------------------
% #1 Filename;  #2 Label;  #3 Bildunterschrift;  #4 Kurzform
\newcommand{\bild}[4]{
  \begin{figure}[htbp]
    \begin{center}
      \includegraphics{#1}
      \caption[#4]{#3}
      \label{#2}
    \end{center}
  \end{figure}
}

% Bildbreite -----------------------------------------------------------------
% #1 Filename;  #2 Breite;  #3 Label;  #4 Bildunterschrift;  #5 Kurzform
\newcommand{\bildbreite}[5]{
  \begin{figure}[htbp]
    \begin{center}
      \includegraphics[width=#2]{#1}
      \caption[#5]{#4}
      \label{#3}
    \end{center}
  \end{figure}
}


% ============================================================================
\begin{document}

% =========== Das war der Vorspann, jetzt geht's los! ========================

% ============================================================================
% =============  AB HIER DARF UND SOLL GETIPPT WERDEN ========================
% ============================================================================

\author{Tobias Schiffmann}
\index{Viel Schreiber}

% Das Institut wird fuer den Betreuer missbraucht ...
\institute{{\bf Betreuer:} Gregor Daiß}
\authorrunning{Viel Schreiber}
\title{SIMT/GPGPU - CUDA \& OpenCL}

\maketitle

\thispagestyle{empty}

\begin{abstract}
Ein schöner Abstract. Das ist einfach die Kurzzusammenfassung.
\end{abstract}


\section{Gliederung}
%------------------------------------------------------------------------
- Motivation / kurzer einstieg in motivation für GPUs
  1 entworfen für Grafikanwendungen mit sehr großen Datenmengen
    --> im Grafikbereich: hohes Potential an Datenparallelität! --> keine Abhängigkeiten
    ==> heutiges Beispiel NNs
  1 sehr hoher FP-Operation durchsatz, verteilt auf große Anzahl Threads
  1 Damals noch mit komplexer Programmierumgebung (DirectX / OpenGL!! G!)
  1 sollen GPU Programmierung vereinfachen
	 --> 4 durch diese vereinfachung ist GPU programmierung zugänglicher und wird für andere Aufgaben als Grafikberechnung ausgeführt --> General Purpose GPU
	   
%------------------------------------------------------------------------  
- GPU - Architektur(en)
    --> Basics des Aufbaus
    --> nur die neuste Architektur
    --> was verwenden Unterschiedliche Hersteller
    1 multi-threaded SIMD-Prozessoren (NVIDEA: Streaming Multiprocessors [SMX]), können als unabhängige MIMD-Kerne betrachtet werden
    1 jeder SIMD Prozessor hat mehrere SIMD-Funktionseinheiten( jede hat Int und FP - Einheit)
    (1 Fermi und Kepler architecture)
    3 Turing architecture !
    
%------------------------------------------------------------------------   
  - Programmier Frameworks (CUDA / OpenCL)	
    --> beide sehr ähnlich --> erst CUDA erklären und dann Unterschiede zu OpenCL eingehen
    1 beide: Program Separierung in Host-Program(CPU [IO + User]) und Device-Program (GPU)

    
    1 Interaktion: Host-Programm kopiert Daten in GPU Speicher und Device Funktionen aufrufen
    
    - CUDA
      1 NVIDIA
      1 device program = kernel-functions/kernels
      1 Program startet mit host-Programm bis Kernel-Funktionsaufruf -> Kernel und Host laufen parallel
      	- Aufruf erzeugt CUDA-Threads (zusammengefasst als Grid)
      CUDA ist nutzbar in verschiedenen PLs (C/C++/Fortran/Python)
      1 erklärungen für C sind hier drin
	  1 Kernelaufruf enthält \textit{execution configuration}
	  	--> gibt die organisation der generierten Threads im Grid der Kernel-Funktion an
	  	  - unterteile Threads des Grids in Blöcke von Threads (3-dim, blockIDx.xyz)
	  	  - Blöcke sind in threads aufgeteilt (3-dim, threadIDx.xyz)
	   --> jeder einzelne Thread des Grids kann aufgerufen werden eine Kernel-Funktion auszuführen      
          - um Threads zu synchronisieren gibt es synchronisationsfunktionen nur innerhalb eines Blocks
      1 verschiedene Speichertypen
        - global memory(host-RW, Device-RW)
        - constant memory (host-RW, device-R)
        - register(nur thread kann drauf zu greifen) and shared memory(alle threads eines Blocks) (kurze Zugriffszeit)
        --> Schaubild !! (1)
      1 Thread Scheduling
        - passiert nicht auf der Ebene einzelner Threads, sondern "Warps" --> mehrere Threads eines Blocks (aufsteigende threadIDx) zusammengefasst
        - eine Instruktion nach der anderen für alle Threads eines Warps ausführen --> \textbf{SIMT}
          --> Instruktionen auf verschiedenen Kontrollflüssen, wenn gleich, dann kann schnell sein
          	Falls, unterschiedlich SIMT nur beschränkt möglich, kann sich auch bei if-else aufteilen
        
        - SIMT
          - 
          2 introduced here !!!
        
      
    - OpenCL
      1 Mehrere partner (u.a. NVIDIA) 
      1 standardisiertes Programmiermodel
      1 heterogene zugrunde liegende Plattform muss explizit spezifiziert werden ("Kontext")
        --> etwas komplexer als CUDA, aber Vielfalt diverser HW möglich
      1 Work-Items (CUDA-Threads) < Work-Groups < globaler NDRanges (Grid)
%------------------------------------------------------------------------
- Vergleich von CUDA, OpenCL und (MIMD oder Sequentiell)
  - Beispiel für hohe Datenparallelität
  	1 hier werden einige Beispiele gennant
  - Beispiel mit geringer Datenparallelität
  --> Schaubilder für execution time aller Beispiele
%------------------------------------------------------------------------
- Conclusion / Discussion

\begin{enumerate}
\item \cite{Rauber.2012} (1)
\item \cite{Lindholm.2008} (2)
\item \cite{Burgess.2020} (3)
\item \cite{Huang.2008} (4)
\item \cite{Bialas.2016} (5)
\item \cite{Khronos.2019} (6)
\item \cite{Wang.2019} (7)
\end{enumerate}



% Einleitung -----------------------------------------------------------------
\section{Einleitung}
Lorem ipsum dolor sit amet, consetetur sadipscing elitr, sed diam nonumy eirmod tempor invidunt ut labore et dolore magna aliquyam erat, sed diam voluptua. At vero eos et accusam et justo duo dolores et ea rebum. Stet clita kasd gubergren, no sea takimata sanctus est Lorem ipsum dolor sit amet. Lorem ipsum dolor sit amet, consetetur sadipscing elitr, sed diam nonumy eirmod tempor invidunt ut labore et dolore magna aliquyam erat, sed diam voluptua. At vero eos et accusam et justo duo dolores et ea rebum. Stet clita kasd gubergren, no sea takimata sanctus est Lorem ipsum dolor sit amet.

\subsection{Anmerkungen zur Einleitung}
Hier kommt noch mehr Text. Wir verweisen dazu auf
%\cite{thisdocument}.

Eine schöne Formel ist
\[
u(\vec{x}) = \sum_{i=1}^N \alpha_i \varphi_i(\vec{x}) \,,
\]
aber das geht auch inline als $u(\vec{x}) = \sum_{i=1}^N \alpha_i
\varphi_i(\vec{x})$, also mitten im Text.

Was noch fehlt ist ein Bild, z.B.\ das aus
Abbildung~\ref{fig:grid1} oder Abbildung~\ref{fig:grid2}. Wir können dazu prima die tollen Makros,
die oben im Vorspann definiert wurden, verwenden. Beispielsweise mit
folgenden Befehlen:
\begin{verbatim}
\bild{figures/grid_l2_brd}{fig:grid1}{Dies ist ein sogenanntes dünnes
Gitter zum Level 2.}{Die Kurzform lasse ich meist leer}
\bildbreite{figures/grid_l2_brd_B}{2cm}{fig:grid2}{Dies ist ein sogenanntes dünnes
Gitter zum Level 2 in 2cm Breite.}{}
\end{verbatim}
Die Bilder werden automatisch nach vernünftigen Kriterien platziert,
daher immer im Text mit \verb!\ref{}! drauf verweisen (bei den
Beispielen mit \verb!\ref{fig:grid1}! und \verb!\ref{fig:grid2}!).
\bild{figures/grid_l2_brd}{fig:grid1}{Dies ist ein sogenanntes dünnes
  Gitter zum Level 2.}{Die Kurzform lasse ich meist leer}
\bildbreite{figures/grid_l2_brd}{2cm}{fig:grid2}{Dies ist ein sogenanntes dünnes
Gitter zum Level 2 in 2cm Breite.}{}

% Anmerkung: damit LaTeX nicht denkt, dass ein Punkt den Satz beendet
% (da spendiert LaTeX gerne mehr Zwischenraum), können wir das
% Leerzeichen mit Backslash als Leerzeichen markieren. Damit LaTeX
% ein Leerzeichen setzt, bei dem es keinen Zeilenumbruch geben darf,
% kann man die Tilde verwenden.
Was wir hin und wieder noch brauchen ist eine Tabelle, wie z.B.\
Tabelle~\ref{tab:irgendwas}.
\begin{table}[htbp]
  \centering
  \caption{Diese Tabelle zeigt nicht die Daten von etwas Sinnvollem,
    sondern einfach irgend etwas. Tabellenbeschriftungen sind oft drüber.}
  \label{tab:irgendwas}
  \begin{tabular}{lrcp{5cm}}
    \toprule
    \multicolumn{3}{c}{Spalten} & Absatz 5cm \\
    \cmidrule(lr){1-3}
    linksbündig & rechtsbündig & zentriert & \\
    \midrule
    1.0 & -1.1 & 1.2 & toller Text, der nach 5cm umbricht und dafür
    brauchen wir einfach mehr Text. \\
    4321.1 & 6543.2 & 7654.3 & mehr Text \\
    2.44 & 4.66 & 6.88 & 8.00 \\
    \bottomrule
  \end{tabular}
\end{table}

\subsection{Quellcode}
Code-Beispiele können mittels \texttt{lstlisting}-Environment eingebunden
werden.
Siehe Listing~\ref{lst:mylisting} als Beispiel.
Alternativen wie \texttt{minted} sind selbstverständlich auch erlaubt, solange
sie Features wie Syntax-Highlighting und Zeilennummern mitbringen.
Code-Beispiele sollten minimal sein, d.h.\ auf den Punkt gebracht und keinen
überflüssigen Code beinhalten.
Es muss standardkonformer Code sein und mit hinzugefügtem Boilerplate-Code
(main, Auslassungen von Überflüssigem, \dots) ohne Fehler compilierbar sein.

Quellcode aus Dateien kann per \texttt{lstinputlisting} einbezogen werden.
Für Inline-Code \texttt{lstinline} verwenden.
Für abstrakte Algorithmen (kein C++-Code) besser eines der algorithm-Packages
verwenden.

\begin{lstlisting}[style=cpp,caption={Example using Lstlisting},label={lst:mylisting},numbers=left]
template <typename T>
struct LessThan {
  bool operator(T a, T b) { return a < b; };
};

std::vector<int> v = { 5, 4, 3, 2, 1 };
std::sort(v.begin(), v.end(), LessThan<int>());
\end{lstlisting}


\subsection{Zum Schluss}
\dots viel Spaß!

% Literaturverzeichnis ------------------------------------------------
\newpage
\bibliographystyle{alphadinLinkLocal}
\bibliography{literatur}

%\iffalse
\end{document}
%\fi
